\documentclass{mcmthesis}
\mcmsetup{CTeX = false,
    tcn = {12833},  %% your team control number
    problem = {B}, %% your chosen problem (A or B)
    sheet = true,
    titleinsheet = True,
    keywordsinsheet = true,
    titlepage = false,
    abstract = false}

\usepackage{newtxtext}%\usepackage{palatino}
\usepackage{comment}
\usepackage{lipsum}
\hypersetup{
    colorlinks=false,
    linkcolor=blue,
    filecolor=blue,
    urlcolor=blue,
    citecolor=cyan,
}
\usepackage{color}
\usepackage{float}
\numberwithin{figure}{section}
\numberwithin{table}{section}
\numberwithin{equation}{section}
\usepackage{enumerate}


\usepackage{placeins}
\usepackage[figuresright]{rotating}
\usepackage[final]{pdfpages}
%\usepackage[
%    backend=bibtex,
%    style=numeric-verb,
%]{biblatex}
\usepackage{babel}

%\addbibresource{citations.bib}


\title{CO\textsubscript{2} and Global Warming}

%! suppress = LineBreak
\begin{document}

    \begin{abstract}
        Abstract

        \begin{keywords}
            Keywords, More Keywords
        \end{keywords}

    \end{abstract}

    \maketitle
    \tableofcontents
    \newpage


    \section{Introduction}

    \subsection{Background}
    The most significant greenhouse gas on Earth is carbon dioxide, which both absorbs and radiates heat. In contrast to oxygen and nitrogen, which together make up the majority of our atmosphere, greenhouse gases absorb heat emitted from the Earth's surface and re-emit it in all directions, including back toward the planet's surface. The natural greenhouse effect that keeps the Earth's atmosphere above freezing would be insufficient without carbon dioxide. People are accelerating the natural greenhouse effect and raising the earth's temperature by releasing more carbon dioxide into the atmosphere. The NOAA Global Monitoring Lab found that in 2021, carbon dioxide accounted for nearly two thirds of the total heating influence of all greenhouse gases created by humans.

    Prior to the Industrial Revolution, carbon dioxide in the atmosphere was consistently around 280 parts per million (ppm).
    The concentration of CO2 in the atmosphere reached 377.7 ppm
    in March of 2004, resulting in the largest 10-year average
    increase up to that time.[1] According to scientists from National
    Oceanographic and Atmospheric Administration (NOAA) and
    Scripps Institution of Oceanography (SIO) the monthly mean
    CO2 concentration level peaked at 421 ppm in May 2022.[2] An
    Organisation for Economic Co-Operations and Development
    (OECD) report predicts a CO2 level of 685 ppm by 2050.[3]


    \subsection{Problem Analysis}
    Expansion on Problem

    \noindent\textbf{Problem one}: Specifics

    \noindent\textbf{Problem two}: Specifics

    \subsection{Keyword Definitions}
    \noindent\textbf{Term 1}: Definition

    \noindent\textbf{Term 2}: Definition

    \noindent\textbf{Term 3}: Definition

    \subsection{Assumptions and Justifications}
    \noindent\textbf{Assumption 1}: Statement
    \textbf{Justification}: blah blah

    \noindent\textbf{Assumption 2}: Statement
    \textbf{Justification}: blah blah

    \noindent\textbf{Assumption 3}: Statement
    \textbf{Justification}: blah blah


    \section{Modeling}
    Introduction of model, similar to abstract

    \subsection{Variables and Parameters}
    See table 2.1:
    \begin{table}[h!]
        \centering
        \begin{tabular}{cc}
            \toprule
            Variable & Definition      \\
            \midrule
            $x$      & description     \\
            $y$      & description     \\
            $z$      & description     \\
            \bottomrule
        \end{tabular}
        \caption{Variables in the Model}
        \label{tab:my_label}
    \end{table}

    \subsection{Model 1}
    blah blah

    \subsection{Model 2}
    blah blah


    \section{Results}
    \noindent (Full data in References.)

    blah blah


    \section{Model Analysis}

    \subsection{Parameter Sensitivity}
    blah blah

    \subsection{General Evaluation}

    \noindent\textbf{Strength 1}: asdf

    \noindent\textbf{Strength 2}: asdf

    \noindent\textbf{Weakness 1}: asdf

    \subsection{Possible Improvements}
    blah blah


    \newpage


    \section{References}

    \subsection{Program Code}
    \noindent Result data generated:
    \begin{verbatim}
    text data stuff

    \end{verbatim}

    \noindent Python program code:
    \begin{lstlisting}[language=Python]
        # pass

    \end{lstlisting}

    \subsection{Bibliography}
%    \printbibliography

\end{document}
